\documentclass{article}
\usepackage{inconsolata}
\usepackage[tt=false]{libertine}
\usepackage{microtype}
\usepackage{hyperref}
\usepackage[intoc]{nomencl}
\makenomenclature

% Graph
\usepackage{dot2texi}
\usepackage{tikz}
\usetikzlibrary{shapes,arrows}
% \usepackage{indentfirst}

\begin{document}
\title{Kotlin Compiler Reading Notes}
\author{Yuxiang Jiang \thanks{yxjiang@linkedin.com}}
\maketitle

% Commands
\newcommand{\textSafeTo}{\texorpdfstring{$\to$}{to} }
% End commands

\tableofcontents

\section{Methodology}
It's often daunting to read through huge and complex codebase like \href{https://github.com/JetBrains/kotlin}{the Kotlin compiler}! Fortunately we have great tools at hand to deal with such complexity. In particular, IntelliJ Idea provides many priceless code navigation tools:
\begin{itemize}
    \item Type Hierarchy (under Navigate)
    \item Structure (under View - Tool Windows)
    \item Find Usage
    \item Breakpoints
\end{itemize}

With these we'll be able to understand the compiler architecture and internals bit by bit...

\begin{dot2tex}[dot,scale=0.3]
\input{arch-diagram.dot}
\end{dot2tex}

\subsection{Building From Source}
Once you have the repo checked out, run \texttt{./gradlew dist} to build everything. This can take 10-20 minutes. Then you will be able to run tests, compiler.cli.cli-runner (i.e. the script runner), and compiler.preloader (i.e. the CLI compiler loader).

\subsection{Tracing Compiler Execution}

It's often useful to use the debugger to understand how the compiler pipeline works. This requires the ability to run custom code and to attach a debugger to the compiler process. One trick that I used was to add a local file to the cli-runner package (or compiler.preloader). This allows me to run or debug arbitrary code there while being able to link the code with the compiler. And since these these packages don't actually have source dependency on the whole compiler (only link to the dist jar), rebuilding is quite fast.

Another way is to link your code with a prebuilt compiler (e.g. add dependency \texttt{"org.jetbrains.kotlin:kotlin-compiler:1.3.70"}). I can't seem to fetch its sourceJar though...

\section{Tl;dr: Pipeline}
\begin{itemize}
    \item Kotlin source
    \item KtElement (Psi)
    \item core.descriptor (ClassicTypeSystemContext)
    \item FirElement (nodes) / FirBasedSymbol (infotable?) (ConeTypeContext)
    \item IrElement (nodes) / IrSymbol (infotable?) (IrTypeSystemContext)
\end{itemize}

% Make this a graph

\subsection{Pipeline for \texttt{kotlin} Script Runner}
Note that the runner uses coroutine in its eval methods and therefore can be hard to trace.

Runner: AbstractScriptEvaluationExtension \textSafeTo ScriptJvmCompilerImpls
Parse: jvmCompilationUtil.getScriptKtFile: text-based Kt source \textSafeTo KtFile (Psi-based KtElement)
Analysis: TopDownAnalyzerFacadeForJVM \textSafeTo LazyTopDownAnalyzer
Codegen (Psi-based): KotlinCodegenFacade \textSafeTo DefaultCodegenFactory \textSafeTo THINGCodegen (e.g. Package / Member / Script / ImplementationBody / ClassBody)
Eval: AbstractScriptEvaluationExtension \textSafeTo BasicJvmScriptEvaluator

\subsection{Pipeline for \texttt{kotlinc} Compiler}

 Runner: compiler.preloader / Preloader \textSafeTo compiler.cli / K2JVMCompiler \textSafeTo KotlinToJVMBytecodeCompiler (checks IR flag)
Parse: Not sure
Analysis: KotlinToJVMBytecodeCompiler.analyze \textSafeTo TopDownAnalyzerFacadeForJVM
Codegen (Psi-based): KotlinToJVMBytecodeCompiler.generate \textSafeTo KotlinCodegenFacade
Lowering (Ir-based): analysis is the same; but ktjvmbcc.generate is different? Uses JvmIrCodeGenFactory with a PhaseConfig. I.e., .generate \textSafeTo JvmIrCodeGenFactory \textSafeTo JvmBackendFacade \textSafeTo JvmLower \textSafeTo CompilerPhase.invokeToplevel(PhaseConfig, JvmBackendContext, IrModuleFragment) \textSafeTo a bunch of abstraction layers around lowering phases \textSafeTo SomeLoweringPass.lower
Codegen (Ir-based): JvmBackendFacade \textSafeTo ClassCodeGen (IrFile.declarations should only contain IrClass after lowering)
Write: KotlinToJVMBytecodeCompiler.writeOutputs

\subsection{Source \textSafeTo KtElement}

\begin{itemize}
    \item See class KtVisitor for an overview of many Kotlin PsiElements
    \item compiler.psi defines KtElement and \href{https://www.jetbrains.org/intellij/sdk/docs/basics/indexing_and_psi_stubs/stub_indexes.html}{Stubs}, also provides parser
    \begin{itemize}
        \item Stubs represents the interface parts of a kt compilation unit
        (.h, .mli, .hi etc).
        \item There's also a LighterASTNNode (KotlinLightParser) -- what is it? (Flyweight pattern is like hash consing)
        \item Some KtElements are related to types: e.g. KtTypeReference (some of them are not KtElements but stubs!)
    \end{itemize}
    \item compiler.psi.KtPsiFactory is the entrypoint for creating KtFile (a PsiElement) from source text.
\end{itemize}

\subsection{KtElement \textSafeTo DeclarationDescriptor}

DeclarationDescriptor seems to be a high-level IR with type analyzed. It contains both the expr tree and the types.

\subsubsection{LazyTopDownAnalyzer}

compiler.frontend.LazyTopDownAnalyzer seems to be the psi analyzer. (Is there another analyzer that's not lazy?)

analyzeDeclarations returns an AnalysisResult which contains a ModuleDescriptor and a BindingContext.

\subsection{DeclarationDescriptor \textSafeTo FIR}

XXX: Not sure if this step is required or even exists.

FIR (compiler.fir) seems to be an intermediate yet still high-level IR.

\begin{itemize}
    \item See generated class FirVisitor for an overview of many of the Fir exprs
    \item compiler.frontend (not sure which step it is in, but it at least does symbol resolution, type checking, (Psi \textSafeTo CFG?)
    \begin{itemize}
        \item See key classes: AnalysisResult, BindingContext, BindingTrace (records the collected binding / type substs?)
    \end{itemize}
    \item compiler.resolution: tower/ReceiverValue etc
    \item compiler.fir: cones (types and symbols used in Fir?), fir2ir (lowering to Ir), psi2fir (lowering Psi to Fir), resolve, jvm, tree (Fir definitions and impl for psi2fir)
    \begin{itemize}
        \item cones: StandardClassIds contains a bunch of core Kt (read: not JVM) type Ids. They have a JVM-like fqname.

        SyntheticCallableId contains when/try/nullcheck synthetic call exprs

        \item tree.gen contains all Fir expressions (see tree.tree-generator's readme for how they are generated), as well as extra info (FirTypeRef). And even on Fir level, the generic types are not yet erased (FirTypeProjectionWithVariance)
    \end{itemize}
\end{itemize}

\subsection{FIR \textSafeTo IR}

XXX: Not sure if this step is required.

\subsection{Example stacktrace for running a Kotlin script}

E.g. \texttt{kotlin -e <expr>}

\begin{itemize}
    \item CLIDriver
    \begin{itemize}
        \item compiler.cli / K2JVMCompiler
        \item plugins.scripting-compiler
        \item compiler.cli / TopDownAnalyzerFacadeForJVM (analyzeFilesWithJavaIntegration)
    \end{itemize}
    \item Analyzer
    \begin{itemize}
        \item compiler.frontend / LazyTopDownAnalyzer.analyzeDeclarations
        \item (HUGE) go through all stmts
        \item BodyResolver.resolveBodies
        \item DeclarationChecker.process
        \item (HUGE) go through files, annotations, class's modifiers, idents, header (super+generic bounds); function, property, destructionDecl, typealias's modifiers and idents.
    \end{itemize}
\end{itemize}

\section{KtElement}

KtFile and KtScript for toplevel decls,
KtClass for \texttt{class Foo}, KtNamedFunction for \texttt{fun foo()}, KtProperty for \texttt{val foo}.

\section{DeclarationDescriptor}
\label{sec:descriptor}

See core.descriptors.DeclarationDescriptorVisitor for an overview.

DeclarationDescriptorVisitor sounds like an type-instantiated/abstracted wrapper of an element. Also has a bunch of annotations (AnnotationDescriptor) and a name. And is also a tree node (has parent: getContainingDecl)

\subsection{Survey of Class Hierarchy}

Looks that descriptors are something that's used throughout the whole compilation pipeline. They are more often used in frontend, but even in backend I can see some usage of them.

\subsubsection{CallableD: VisD \& NonRootD \& Subst}

Receiver types (dispatch / extension), arg types, return types, type params;

Parameter names, names may be unstable/synthesized (e.g. from JVM object code)

Parameter values (See \ref{sec:ValueParamD})

Cross ref to overridden methods

UserDataKey$<$A$>$: stores typed user data

\subsubsection{MemberD: VisD \& NonRootD}

Has member modifiers: expect / actual / external. And modality: final /sealed / open / abstract.

\subsubsection{CallableMemberD: CallableD \& MemberD}

Kind: decl / delegation / fakeOverride / synthesized (what's the last two?)

\subsubsection{ValueD: CallableD}

Has a KotlinType.

\subsubsection{VarD: ValueD}

Has isVar (wat), isLateinit, isConst, and an optional compileTimeInitializer.

\subsubsection{ParamD: ValueD}
Represents a parameter that can be supplied to a callableD.

\paragraph{ReceiverParamD: ParamD}
Has a ReceiverValue.

\paragraph{ValueParamD: ParamD \& VarD}
\label{sec:ValueParamD}
Has a index, hasDefaultValue, varargElementType, isCrossinline / Noinline (why on param?)

\subsubsection{VarDWithAccessors}

Has optional getter / setter typed VarAccessorD.

\subsubsection{FunD: CallableMemberD}

initialSignatureD: the initial D before renaming (didn't find SimpleFunctionD.rename)

hiddenToOvercomeSignatureClash: hack to handle corner case signature clash (said see nio.CharBuffer); also hiddenEverywhereBesideSupercalls: see \ref{sec:FunDImpl-hidden}.

Function modifiers: infix/inline/operator/suspend/tailrec

\subsubsection{FunDImpl: NonRootDImpl \& FunD}

Base impl for function modifiers. Setters set the local modifier (mostly happen during conversion from KtElement), while some getters (infix, operator) respect super class methods.

Base impl for substitution (doSubstitute), and substituted value param. Worth reading.

Base impl for initialize.

\label{sec:FunDImpl-hidden}
Only here documents hiddenToOvercomeSignatureClash and hiddenEverywhere-BesideSupercalls: former makes the function completely hidden (even in super-call), latter permits super-call and propagates to overriden methods

\subsubsection{ConD: FunD}

containingD: ClassifierDWithTypeParams (what is this?)

constructedClass: ClassD

\subsubsection{ClassConD: ConD}

Just a bunch of return type specializations

\subsubsection{ClassConDImpl: FunDImpl \& ClassConD}

Default (\texttt{<init>}) or synthesized.

Has a way to calculate dispatchReceiverParam. If inner, init's receiver is outer class instance (Whereas Java's outer class instance is passed as a param. Though in compiled code there's no difference, just at descriptor level they are different.); else null.

\paragraph{CommonizedClassConD: ClassConDImpl}
ClassConDImpl with source and originalD stripped.

\paragraph{DefaultClassConD: ClassConDImpl}
Read as (default constructor) of given class, not default implementation of (any class constructor). So this is just the no-param constructor of a class. Its visibility depends on the classD's visibility.

\paragraph{DeserClassConD: ClassConDImpl \& DeserCallableMemberD}
ProtoBuf based deserialized ClassConD. Holds a bunch of TypeTable, NameResolver, ContainerSource etc to help with further deserialization (so this is also in some sense lazy).

\paragraph{JClassConD: ClassConDImpl \& JCallableMemberD}
A Java imported class's constructor. Since this is from JVM object code, it provides property impl of hasStableParamNames / hasSynthesizedParamNames.

enhance() implementation makes a copy with enhanced receiver param and value params.

\paragraph{SamAdapterClass: ClassConDImpl \& SamAdapterD$<$JavaClassConD$>$}
A synthetic ClassConD that wraps another ClassConD.

\subsection{DFactory}

In core.descriptors / resolve.

\section{FIR}
\label{sec:fir}

Front-end IR (not sure what this means), enabled by CommonConfigurationKeys .USE\_FIR (This also implies using IR. But use-ir doesn't imply using FIR).

compiler.fir.resolve: ResolutionStage sounds like something pipeline-related

\section{IR}
\label{sec:ir}

IR (compiler.ir) seems to be an lower-level IR. This is an experimental IR that's intended to be used across all Kotlin backends. Also see \href{https://medium.com/@bnorm/exploring-kotlin-ir-bed8df167c23}{this writeup}.

A sample IR lowering pass can be found \href{https://github.com/JetBrains/kotlin/blob/936e53d/compiler/ir/backend.common/src/org/jetbrains/kotlin/backend/common/lower/TailrecLowering.kt}{here}. Looks that the compiler API provides a convenient IrBuilder, and a visitor style IrTransformer.

\subsection{Playing with IR}
This is enabled by passing \texttt{-Xuse-ir} as a CLI option to the compiler. IR backend currently doesn't support the script runner -- KotlinToJVMBytecodeCompiler will ignore use-ir on kts files; Also ScriptCodegen calls KotlinTypeMapper.mapType which is not intended to be called with IR backend.

\subsection{compiler.ir.tree}
IrSymbol definitions. See IrSymbolVisitor for an overview. Looks that they have descriptors attached.

\subsection{Phases}
compiler.ir.backend.common defines CompilerPhase that's chainable. See PhaseBuilders.kt and JvmLower.kt.

\subsection{IrSymbol vs IrElement}
IrElement seems to be the implementation part while IrSymbol is more about the declaration part. E.g. IrClassImpl vs IrClassSymbolImpl. The former contain the concrete class members (init, methods) while the latter contains ClassD.

IrSymbol can be bound or not (what does this mean?), has a owner (an IrElement), an IdSignature (what's this?), and visibility (isPublicApi).


\section{Type System}

\subsection{Types}

Kotlin compiler uses \href{https://en.wikipedia.org/wiki/Marker_interface_pattern}{Marker Interface pattern} extensively in type definitions. See TypeSystemContext.kt (KotlinTypeMarker, TypeArgumentMarket etc).

KotlinType is the base class for all types in Kotlin. It has a tycon, list of tyargs, nullability (so nullability is built-in to any type -- can this be problematic?). It also has a MemberScope ("what are the members in this type-based namespace?", see \ref{sec:reso}) and a refine(KotlinTypeRefiner) method.

\subsection{core.type-system}

type system core (equality, bounds checking etc). However this is more of an interface module -- the actual impls are in core.descriptors, fir and ir modules.

\subsection{TypeSystemTypeFactoryContext}

Contains a bunch of common type factories:

\begin{itemize}
    \item flexibleType has lower/upper bounds
    \item simpleType has tycon, tyargs, nullablep
    \item tyarg has ty and variance
    \item star has tyarg (why?)
    \item there's also an error type used in diagnosis
\end{itemize}

\subsection{TypeCheckerProviderContext}

\begin{itemize}
    \item modular axioms (errorType unifiable with all types etc)
    \item what is a stub type? (Probably PsiStub related?)
\end{itemize}

\subsection{TypeSystemCommonSuperTypesContext}

Used to check if two type has common super types, and lowest-common ancestor utils.

typeDepth is a safe overestimation of the depth (from `Any`).

Seems to also be used in Fir.

\subsection{TypeSystemInferenceExtensionContext}

Inference related.

\subsubsection{Questions}
\begin{itemize}
    \item What is a isCapturedTypeConstructor?
    \item What is a singleBestRepresentative?
    \item What is a noInferAnnotation?
    \item What is mayBeTypeVariable?
    \item What is a defaultType?
    \item Read impl of isUnit vs isUnitTypeConstructor
    \item Read impl of createCapturedType
    \item Read impl of createStubType
    \item Read impl createEmptySubstitutor, typeSubstitutorByTypeConstructor, safeSubstitute
\end{itemize}

\subsection{TypeSystemContext}

\begin{itemize}
    \item fastCorrespondingSupertypes has no actual impl? (No, it's just that intellij's search functionality fail to find overridden extension methods)
    \item isCommonFinalClassConstructor is implemented in three (Psi, Fir, Ir) stage's TypeSystemContext:
    \begin{itemize}
        \item ClassicTypeSystemContext: get ClassDescriptor from TypeConstructor's declarationDescriptor, then check it's final but not (enum or annotation). So the method really checks that the tycon is "final" but is not a uncommon (enum/annotation) class.
        \item ConeTypeContext: Does almost the same thing, but also return true if is anonymous object (final by design). Works on FirBasedSymbol (some sort of class infotable?). Check that this is a FirRegularClassSymbol, whose FirRegularClass is final but not uncommon.
        \item IrTypeSystemContext: Check this is a IrClassSymbol whose owner is final and not uncommon.

        So basically ClassDescriptor, FirRegularClass and IrClassSymbol.owner are the same thing across three stages.

        Sounds that reading the common implemented methods of these three TySysCtx impl classes would be super helpful to understand the stages.
    \end{itemize}
\end{itemize}

\subsection{Type System for Fir}

Read ConeTypeContext

\subsection{Type System for Ir}

Read IrTypeSystemContext

\subsection{compiler.resolution/.inference}

Type inference? constraint system, subst, fresh tycon, tyvar etc

\subsection{compiler.frontend/.types}

TypeIntersector (unify), DeferredType (I guess this is for when inference can't proceed at some first, and will retry when it has more information. Not really fully bidirectional (H-M style) type inference, but an approximation)

\subsection{types.expressions}

Contains a bunch of KtElement visitors that does type recon/checking:
\begin{itemize}
\item ExpressionTypingVisitorDispatcher
\item ControlStructureTypingVisitor
\item FunctionsTypingVisitor
\item BasicExpressionTypingVisitor (constants etc)
\begin{itemize}
  \item This actually does a bit of parsing/validation work... e.g. understore on int literals.
  \item Also uses ConstantExpressionEvaluator to check for possible compile time constants (this indeed sounds like something a parser would do).
  
  Folds boolean \texttt{\&\&} and \texttt{||}

  Look up simple unary and binary func in OperationsMapGenerated
\end{itemize}
\end{itemize}

\section{Resolution}
\label{sec:reso}

Package: compiler.resolution

\subsection{Scopes}

core.descriptor / ResolutionScope: contains information about what identifier it contributes to a given lookup location. Identifiers have separate namespaces: variable, function, and classifier (type).

compiler.resolution / LexicalScope is a ResolutionScope that has a parent, a ownerD, a LexicalScopeKind (what kind of syntactical structure created this scope?)

\subsection{Tower}

i.e. ImplicitScopeTower. Some sort of multi-level scopes? Can't understand this part.

\section{Analysis}

Package: compiler.frontend

\subsection{compiler.frontend.LazyTopDownAnalyzer}

\subsubsection{TopDownAnalysisContext}

Stores the toplevel declarations (in typed maps) found during analysis.

\subsection{BindingTrace}

Has a BindingContext. Is writable. Can record/inquiry KotlinType for a KtElement.
Impls: BindingTraceContext and ObservingBindingTrace

\subsection{BindingContext}

Sounds like a read-only counterpart to the BindingTrace.

\subsection{Smartcasting}

compiler.frontend smartcasts.DataFlowInfo: bunch of maps to stores the data flow analysis result useful for smart casts.

DataFlowValue: one instance of a value in a dataflow

DataFlowValue.Kind: classify exprs into smart cast enabled, possible, or disabled ones. Quite intuitive.

IdentifierInfo: represents both qualifier and ident name. what is this for?

\section{Codegen}

There are currently two codegen systems and both target JS, JVM, etc. They differ in the choice of IR.

\begin{itemize}
    \item The existing production codegen uses KtElement (Psi node) as the IR. JVM codegen lives in compiler.backend / codegen (e.g. ClassBodyCodeGen); JS codegen lives in js.js.translator (e.g. PropertyTranslator).
    \item The other "experimental" codegen uses compiler.ir as the IR. All targets specific code lives in compiler.ir.backend.TARGET.
\end{itemize}

\subsection{Psi-based Codegen}

\subsubsection{JVM}

\paragraph{ExpressionCodeGen}
Generates JVM bytecode directly from KtElement. This is a KtVisitor$<$StackValue$^2$$>$. Has an InstructionAdapter (JVM bytecode emitter).

\paragraph{StackValue}
Sounds like a helper class to represent operands on the JVM stack. Has a couple subclasses.

\paragraph{ClassBodyCodeGen}
Generate class body from a KtPureClassOrObject with a ClassDescriptor. Also does bridge generation.

\paragraph{FunctionCodeGen}
Generate class body from a KtPureClassOrObject with a ClassDescriptor. Also does bridge generation.

\subsubsection{JS}

e.S. js.js.translator / PropertyTranslator. Looks that it directly translates descriptor + Psi (KtElement) into JS statement / expressions.

Also see FunctionBodyTranslator -- takes a FunD and a KtElement (KtDeclWithBody).

\subsection{IR-based Codegen}

\subsubsection{JVM}

\paragraph{ClassCodegen}
Takes an IrClass and generates into a JvmBackendContext.

\section{Unsolved Questions}

\begin{itemize}
    \item What is a LazyTypeParameterDescriptor?
    \item What is a core.descriptors.MemberScope?
    \item What's the diff between TypeCheckerContext and TypeSystemContext?
\end{itemize}

\nomenclature{Fun}{Function}

\nomenclature{Con}{
Constructor. Can be either a type constructor like $\forall$ \texttt{a, List<a>}, or a value / class constructor like \texttt{<init>}}

\nomenclature{Vis}{
Visibility, as in public / private / internal etc.}

\nomenclature{Subst}{
Substitution, usually in the context of type checking / reconstruction / unification algorithms. See \href{https://en.wikipedia.org/wiki/Unification_(computer_science)}{Wikipedia: Unification} and \href{https://ncatlab.org/nlab/show/substitution}{nLab: Substitution}.}

\nomenclature{D}{
DeclarationDescriptor or just Descriptor. See the \hyperref[sec:descriptor]{DeclarationDescriptor section}}

\nomenclature{J}{Java}

\nomenclature{Impl}{Implementation}

\nomenclature{FIR}{Intermediate level IR, below KtElement and above IR. See package compiler.fir.}
\nomenclature{IR}{Low level IR. See package compiler.ir.}

\nomenclature{Commonization}{Seems to be a klib process to strip source and original information from descriptors. For example see CommonizedClassConD. Maybe this is to reduce storage size for these Ds in klibs?}

\nomenclature{Deser}{Deserialization. See core.deserialization -- The compiler is able to serialize / deserialize descriptors. And it seems to be done lazily -- Many deserialized descriptors still hold TypeTable / NameResolver / VersionRequirementTable to help with further deserialization.}

\nomenclature{PSI}{JetBrain's universal parse tree API. See \href{https://www.jetbrains.org/intellij/sdk/docs/basics/architectural_overview/psi.html}{its doc here}.}

\nomenclature{Stub}{In the context of PSI, this is the interface part of a PSI tree. It's initially calculated from PSI trees and then cached for efficient retrival. See \href{https://www.jetbrains.org/intellij/sdk/docs/basics/indexing_and_psi_stubs/stub_indexes.html}{JetBrain's doc}.}
\newpage
\printnomenclature

\end{document}
