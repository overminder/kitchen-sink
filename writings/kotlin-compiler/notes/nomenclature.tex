% Language general abbrevs

\nomenclature{Fun}{Function}

\nomenclature{Con}{
Constructor. Can be either a type constructor, or a value / class constructor like \texttt{<init>}}

\nomenclature{TyCon}{\href{https://en.wikipedia.org/wiki/Type_constructor}{Type constructor}. Basically a type-level function that takes zero or more types, and returns another type. E.g. List is a tycon: $\forall$ \texttt{a : Type, List<a> : Type}}

\nomenclature{Subst}{
Substitution, usually in the context of type checking / reconstruction / unification algorithms. See \href{https://en.wikipedia.org/wiki/Unification_(computer_science)}{Wikipedia: Unification} and \href{https://ncatlab.org/nlab/show/substitution}{nLab: Substitution}.}

\nomenclature{Impl}{Implementation}

\nomenclature{Param}{Parameter}

\nomenclature{Var}{Variable}

% JetBrains specific abbrevs

\nomenclature{PSI}{JetBrain's universal parse tree API. See \href{https://www.jetbrains.org/intellij/sdk/docs/basics/architectural_overview/psi.html}{its doc here}.}

\nomenclature{Stub}{In the context of PSI, this is the interface part of a PSI tree. It's initially calculated from PSI trees and then cached for efficient retrival. See \href{https://www.jetbrains.org/intellij/sdk/docs/basics/indexing_and_psi_stubs/stub_indexes.html}{JetBrain's doc}.}

% Kt specific abbrevs

\nomenclature{J}{Java}

\nomenclature{Vis}{
Visibility, as in public / private / internal etc.}

\nomenclature{D}{
DeclarationDescriptor or just Descriptor. See the \hyperref[sec:descriptor]{DeclarationDescriptor section}}

\nomenclature{FIR}{"Front-end IR", an intermediate level IR, below KtElement and above IR. See \ref{sec:fir}.}

\nomenclature{IR}{Can either be a general term (Intermediate Representation), or specifically Kotlin compiler's new experimental lowlevel IR (see \ref{sec:ir}).}

\nomenclature{Commonization}{Seems to be a klib process to strip source and original information from descriptors. For example see CommonizedClassConD. Maybe this is to reduce storage size for these Ds in klibs?}

\nomenclature{Deser}{Deserialization. See core.deserialization -- The compiler is able to serialize / deserialize descriptors. And it seems to be done lazily -- Many deserialized descriptors still hold TypeTable / NameResolver / VersionRequirementTable to help with further deserialization.}

\nomenclature{Bridge}{Kotlin's way (i.e. doesn't require Java 8) of representing default implementation for interfaces in generate code. See compiler.backend-common / impl.kt's code doc.}